\documentclass{scrartcl}

\usepackage[francais]{babel} 
\usepackage[utf8]{inputenc}  %% les accents dans le fichier.tex

\usepackage{times}
\usepackage[T1]{fontenc}       %% Pour la césure des mots accentués
\usepackage{graphicx}
\usepackage{url}
\usepackage{xspace}
\usepackage{listings}
\usepackage[export]{adjustbox}
\usepackage{caption}


\newcommand{\code}[1]{\texttt{#1}}

\lstdefinelanguage{CSS}{
      keywords={accelerator,azimuth,background,background-attachment,
            background-color,background-image,background-position,
            background-position-x,background-position-y,background-repeat,
            behavior,border,border-bottom,border-bottom-color,
            border-bottom-style,border-bottom-width,border-collapse,
            border-color,border-left,border-left-color,border-left-style,
            border-left-width,border-right,border-right-color,
            border-right-style,border-right-width,border-spacing,
            border-style,border-top,border-top-color,border-top-style,
            border-top-width,border-width,bottom,caption-side,clear,
            clip,color,content,counter-increment,counter-reset,cue,
            cue-after,cue-before,cursor,direction,display,elevation,
            empty-cells,filter,float,font,font-family,font-size,
            font-size-adjust,font-stretch,font-style,font-variant,
            font-weight,height,ime-mode,include-source,
            layer-background-color,layer-background-image,layout-flow,
            layout-grid,layout-grid-char,layout-grid-char-spacing,
            layout-grid-line,layout-grid-mode,layout-grid-type,left,
            letter-spacing,line-break,line-height,list-style,
            list-style-image,list-style-position,list-style-type,margin,
            margin-bottom,margin-left,margin-right,margin-top,
            marker-offset,marks,max-height,max-width,min-height,
            min-width,-moz-binding,-moz-border-radius,
            -moz-border-radius-topleft,-moz-border-radius-topright,
            -moz-border-radius-bottomright,-moz-border-radius-bottomleft,
            -moz-border-top-colors,-moz-border-right-colors,
            -moz-border-bottom-colors,-moz-border-left-colors,-moz-opacity,
            -moz-outline,-moz-outline-color,-moz-outline-style,
            -moz-outline-width,-moz-user-focus,-moz-user-input,
            -moz-user-modify,-moz-user-select,orphans,outline,
            outline-color,outline-style,outline-width,overflow,
            overflow-X,overflow-Y,padding,padding-bottom,padding-left,
            padding-right,padding-top,page,page-break-after,
            page-break-before,page-break-inside,pause,pause-after,
            pause-before,pitch,pitch-range,play-during,position,quotes,
            -replace,richness,right,ruby-align,ruby-overhang,
            ruby-position,-set-link-source,size,speak,speak-header,
            speak-numeral,speak-punctuation,speech-rate,stress,
            scrollbar-arrow-color,scrollbar-base-color,
            scrollbar-dark-shadow-color,scrollbar-face-color,
            scrollbar-highlight-color,scrollbar-shadow-color,
            scrollbar-3d-light-color,scrollbar-track-color,table-layout,
            text-align,text-align-last,text-decoration,text-indent,
            text-justify,text-overflow,text-shadow,text-transform,
            text-autospace,text-kashida-space,text-underline-position,top,
            unicode-bidi,-use-link-source,vertical-align,visibility,
            voice-family,volume,white-space,widows,width,word-break,
            word-spacing,word-wrap,writing-mode,z-index,zoom},  
      sensitive=true,
      morecomment=[l]{//},
      morecomment=[s]{/*}{*/},
      morestring=[b]',
      morestring=[b]",
      alsoletter={:},
      alsodigit={-}
    }
    
    

\lstset{
         keywordstyle=\color{blue}\bfseries,
        commentstyle=\color{darkgray}\ttfamily,
        ndkeywordstyle=\color{editorGreen}\bfseries,
        stringstyle=\color{editorOcher},
}



\begin{document}


\title{Rapport de Site}
\author{Lecarpentier Jeremie
\and Pinçon Antoine}
%\date{8 Octobre 2014}
\maketitle


\tableofcontents
\listoftables
\listoffigures



\section{Intro}


\subsection{Titre, description du contenu}
\begin{itemize}
\item thème : évènements Normands
\item 3 sous thèmes choisis, avec 5 événements au total
\begin{itemize}
\item Festivals musicaux

\begin{itemize}
\item Festival de Beauregard
\item Festival  Jazz sous les pommiers
\end{itemize}
\item Festival de cinéma
\begin{itemize}
\item Festival du cinéma américain de Deauville
\end{itemize}
\item Festivals traditionnels
\begin{itemize}
\item Fetes médiéval de Bayeux
\item Festival Cidre et Dragons
\end{itemize}
\end{itemize}

\subsection{Pourquoi ces choix ?}


\end{itemize}


\subsection{Logiciels et langages utilisés}
HTML5, CSS3, JavascriptX, gimp et photoshop pour l'édition d'images, latex pour le compte rendu (éditeur : teXshop / teXmaker)


\section{Structures du Site}
\subsection{Plan sémantique du site}
figures
\subsection{Structure du menu de navigation}
figures
\subsection{Plan navigationnel du site}
figure
\subsection{Organisation des dossiers}
figure

\section{Qu'avons nous voulu faire ?}
\subsection{Première version en HTML/CSS}
\begin{itemize}
\item premiers schémas
\item éléments permanents (header par exemple)
\item éléments provisoires (slider avec transitions dans le CSS par exemple)
\end{itemize}


\subsection{Ajout de javascript}
\begin{itemize}
\item refonte du "slider"
\item ajout de styles dynamiques
\item ajout ou retrait d'éléments de la page de façon dynamique ?
\item ajout de trucs marrants qui ne servent a rien ?
\end{itemize}


\section{Solutions techniques pour réaliser le site}
\begin{itemize}
\item menus déroulants
\item slider
\item affichage d’un bloc en passant sur un autre
\item mise en page des blocs (schémas, images...)
\item avec à chaque fois des snippets, des HTML tag content is not necessary, CSS, javascript 4- W3C
\item erreurs rencontrées, solutions apportées. 

\end{itemize}

\end{document}



